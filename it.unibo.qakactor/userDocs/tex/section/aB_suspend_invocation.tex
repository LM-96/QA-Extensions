\section{Invocation of \texttt{suspend} methods}

As we said, in the methods marked with \texttt{@State} annotation, the application designer can call some methods like \texttt{emit(...)}, \texttt{send(...)} or \texttt{answer(...)} but according to the \texttt{QAK} specifications these operations are \href{https://kotlinlang.org/docs/coroutines-basics.html#extract-function-refactoring}{\texttt{suspend fun}}.

So, when the \texttt{AnnotationLoader} load all the classes marked with \texttt{@Actor}, it has to consider that some \textit{state methods} can be suspendable, so they must be called from a \href{https://kotlinlang.org/docs/coroutines-guide.html}{\texttt{Coroutine}}.
For this reason, we have developed \href{https://github.com/LM-96/QA-Extensions/blob/main/it.unibo.qakactor/src/main/kotlin/utils/MethodUtils.kt}{\texttt{MethodUtils.kt}}, a small \texttt{.kt} file with some utility methods, in particular we have the function:
\begin{center}
	\begin{tabular}{c}
		\begin{lstlisting}[frame=none,numbers=none,language=Kotlin]
			suspend fun Method.invokeSuspend(obj : Any, vararg param : Any?) : Any
		\end{lstlisting}
	\end{tabular}
\end{center}

This method uses the \href{https://kotlinlang.org/api/latest/jvm/stdlib/kotlin.coroutines.intrinsics/suspend-coroutine-unintercepted-or-return.html}{\texttt{suspendCoroutineUninterceptedOrReturn}} built-in function that obtain the current coroutine \href{https://kotlinlang.org/api/latest/jvm/stdlib/kotlin.coroutines/-continuation/}{\texttt{Continuation}} and call be \texttt{block} passed as parameter inside this continuation.
Summarizing, the \href{https://kotlinlang.org/docs/extensions.html}{extension function} \texttt{invokeSuspend} we have defined is able to obtain the current continuation and invoke the method using the instance which calls the function by using reflection.

Let us make a simple example (see \href{https://github.com/LM-96/QA-Extensions/blob/main/it.unibo.qakactor/src/main/kotlin/demo/InvokeSuspendDemo.kt}{\texttt{InvokeSuspendDemo.kt}}):
\begin{lstlisting}[caption=Example for \texttt{invokeSuspend}, language=Kotlin]
	class ExampleClazz(val exampleName : String) {
		
		suspend fun suspendWelcome(name : String) {
			println("[${exampleName}] Welcome from suspend $name")
		}
	}
	
	fun main(args : Array<String>) {
		val suspendWelcomeMethod = ExampleClazz::class.java.methods
		.find { it.name == "suspendWelcome" }
		val exampleInst = ExampleClazz("EXAMPLE NAME")
		runBlocking {
			suspendWelcomeMethod?.invokeSuspend(exampleInst, "main")
		}
	}
\end{lstlisting}
The example shows how it is possible to use our \texttt{invokeSuspend} function in order to invoke a method using an instance of \href{https://docs.oracle.com/javase/8/docs/api/java/lang/reflect/Method.html}{\texttt{Method}} class (by reflection). 

Notice that \textcolor{BrickRed}{\textbf{at the moment a method marked with \texttt{@State} annotation must be \texttt{suspend}}}. This is a \textit{small} limitation because the developer is forced to make all \textit{state methods} \texttt{suspend}, but it is not a problem and in future developments this mechanism can be improved very rapidly.